\documentclass[a4paper,12pt]{book} % book
\usepackage[utf8]{inputenc}
\usepackage[T2A]{fontenc}
\usepackage[russian]{babel}
\usepackage{amsmath}
\usepackage{fancyhdr}

\textwidth=120mm
\pagestyle{myheadings}
\pagestyle{fancy}
\fancyhf{}
\addtolength{\oddsidemargin}{15mm}
\addtolength{\evensidemargin}{4mm}
\addtolength{\headheight}{3pt}
%\renewcommand{\headrulewidth}{4pt}

\fancyhead[LO]{310}
\fancyhead[RE]{311}
\fancyhead[CO]{Глава 21}
\fancyhead[CE]{Определение интеграла}

\begin{document}
\noindent По теореме 343, примененной к интервалу $\left[{\eta, \frac{\xi + b}{2}}\right]$, производная функции $L(\eta, x)$ при $x = \xi$ существует и равна $f(\xi)$; следовательно,
\begin{displaymath}
g'(\xi) = f(\xi)
\end{displaymath}
\hspace{49mm}\rule{22mm}{0.4pt}

После этого усилия мы "проинтегрируем" ряд знакомых нам непрерывных функций; их интегралами также окажутся знакомые нам функции.

\textbf{Теорема 345.} \textit{Если n} $\neq$ -1, \textit{то}
\begin{displaymath}
\int_{}^{} x^n dx = \frac{x^{n+1}}{n+1} + c \text{ при } a \geq 0
\end{displaymath}
(т. е. в каждом открытом интервале $a < x < b$ с $a \geq 0$).

Доказательство. Для $x > 0$, по теореме 109, имеем\\
\vspace{2mm}(1)\hspace{10mm}$(\frac{x^{n+1}}{n+1})' = \frac{1}{n+1}(x^{n+1})' = \frac{1}{n+1}(n+1)x^n = x^n$ \vspace{2mm}

\textbf{Теорема 346.} Если $n \neq -1$ \textit{и целое, то}
\begin{displaymath}
\int_{}^{} x^n dx = \frac{x^{n+1}}{n+1} + c \text{\textit{ при }} a \geq 0 \text{\textit{ и при }} b \leq 0.
\end{displaymath}

Доказательство. В силу теоремы 119, формула (1) верна для всех $x \neq 0$.

\textbf{Теорема 347.} \textit{Если} $n\geq0$ \textit{и целое, то}
\begin{displaymath}
\int_{}{} x^n dx = \frac{x^{n+1}}{n+1} + c
\end{displaymath}
\textit{для всех x}.

Доказательство. В силу теоремы 103, формула (1) верна для всех \textit{x}.

\textbf{Теорема 348.}
\begin{displaymath}
\int_{}{} \frac{dx}{x} = 
\begin{cases}
log x + c \hspace{8mm}$\textit{при} $ a \geq 0\\
log(-x) + c \hspace{2mm}$\textit{при} $ b \leq 0
\end{cases}
\end{displaymath}

\newpage
\textbf{Предварительные замечания.} 1) Тем самым заполнен пробел $n = -1$ в теоремах 345 и 346. Но само существование интеграла следует уже из теоремы 344.

2) Результат можно объеденить в одну формулу
\begin{displaymath}
\int_{}^{} \frac{dx}{x} = log |x| + c = \frac{1}{2} log(x^2) + c,
\end{displaymath}

\noindentгодную в обоих случаях.

Доказательство: теоремы 104 и 105.

\textbf{Теорема 349.}

\begin{displaymath}
\int_{}^{} \sum_{i = 0}^{n} a_i x^i dx = \sum_{i=0}^{n} a_i \frac{x^{i+1}}{i+1} + c
\end{displaymath}

\textbf{Предварительное замечание.} Таким образом, интеграл от полинома есть полином. Мы уже знали, что производная от полинома есть полином.

Доказательство. Подинтегральная функция есть производная правой части

\textbf{Теорема 350.}
\begin{displaymath}
\int_{}^{} e^x dx = e^x + c
\end{displaymath}

Доказательство.
\begin{displaymath}
(e^x)' = e^x.
\end{displaymath}

\textbf{Теорема 351.}
\begin{displaymath}
\int_{}^{} \frac{dx}{x-\gamma} = log|x-\gamma| + c \text{\textit{ при }} a \geq \gamma \text{\textit{ и при }} b \leq \gamma
\end{displaymath}

Доказательство. Если $x > \gamma$, то
\begin{displaymath}
(log(x-\gamma))' = \frac{1}{x-\gamma}(x-\gamma)' = \frac{1}{x-\gamma};
\end{displaymath}

\noindentесли $x < \gamma$, то
\begin{displaymath}
(log(\gamma - x))' = \frac{1}{\gamma - x}(\gamma - x)' = \frac{1}{x-\gamma}.
\end{displaymath}

\end{document}
